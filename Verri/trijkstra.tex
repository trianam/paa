%This work is licensed under the Creative Commons
%Attribution-ShareAlike 4.0 International License. To view a copy of
%this license, visit http://creativecommons.org/licenses/by-sa/4.0/ or
%send a letter to Creative Commons, PO Box 1866, Mountain View, CA
%94042, USA.

%This work is licensed under the Creative Commons
%Attribution-ShareAlike 4.0 International License. To view a copy of
%this license, visit http://creativecommons.org/licenses/by-sa/4.0/ or
%send a letter to Creative Commons, PO Box 1866, Mountain View, CA
%94042, USA.

%\documentclass[gray,handout, pdftex, 11pt]{beamer}
%\documentclass[handout, pdftex, 11pt]{beamer}

\documentclass[pdftex, 11pt]{beamer}

\usepackage[utf8]{inputenc}
\usepackage[T1]{fontenc}
\usepackage{lmodern}
%\usepackage[italian]{babel}
\usepackage{acronym}
\usepackage{graphicx}
\usepackage{multirow}
\usepackage{listings}
\usepackage{microtype}
\usepackage{acronym}
\usepackage{array}
\usepackage{tikz}
\usetikzlibrary{shapes, chains, scopes, shadows, positioning, arrows,
  decorations.pathmorphing, calc, mindmap, petri}

\def\transW{8mm}
\def\transH{2mm}

\colorlet{mmcb}{black!20}
\colorlet{mmc1}{red!80}
\colorlet{mmc2}{blue!80}
\colorlet{c1}{green!20}
\colorlet{c2}{blue!10}
\colorlet{c3}{yellow!10}
\colorlet{c4}{red!10}
\colorlet{drawColor}{black!80}
\colorlet{commentColor}{green!70!black!90}
\colorlet{codeBgColor}{yellow!50}
\colorlet{bashBgColor}{green!50}

\tikzset{onslide/.code args={<#1>#2}{%
  \only<#1>{\pgfkeysalso{#2}} % \pgfkeysalso doesn't change the path
}}
\tikzset{temporal/.code args={<#1>#2#3#4}{%
  \temporal<#1>{\pgfkeysalso{#2}}{\pgfkeysalso{#3}}{\pgfkeysalso{#4}} % \pgfkeysalso doesn't change the path
}}

\tikzstyle{oval}=[ellipse, align=center, drop shadow, draw=drawColor, fill=white]
\tikzstyle{rect}=[rectangle, rounded corners=2pt, align=center, drop
shadow, draw=drawColor, fill=white]
\tikzstyle{comment}=[text=commentColor,font=\itshape]
\tikzstyle{textLab}=[]
\tikzstyle{arrow}=[->, very thick, >=stealth', draw=black!80]
\tikzstyle{darrow}=[->, dash pattern=on 3pt off2pt, very thick, >=stealth', draw=black!80]
\tikzstyle{fInt}=[circle, align=center, drop shadow, draw=drawColor, fill=white]
\tikzstyle{fStartEnd}=[ellipse, align=center, drop shadow, draw=drawColor, fill=white]
\tikzstyle{fInput}=[trapezium, trapezium left angle=70, trapezium right angle=110,
align=center, drop shadow, draw=drawColor, fill=white]
\tikzstyle{fProcess}=[rectangle, align=center, drop shadow, draw=drawColor, fill=white]
\tikzstyle{fSelection}=[diamond, shape aspect=3, align=center, drop
shadow, draw=drawColor, fill=white]
\tikzstyle{fOutput}=[tape, tape bend top=none, align=center, drop shadow, draw=drawColor, fill=white]
\tikzstyle{mem}=[rectangle, align=center, draw=drawColor, fill=white]
\tikzstyle{clo}=[cloud, aspect=2, align=center, drop shadow, draw=drawColor, fill=white]
\tikzstyle{file}=[chamfered rectangle, chamfered rectangle corners =
north east, align=center, drop shadow, draw=drawColor, fill=white]
\tikzstyle{files}=[chamfered rectangle, chamfered rectangle corners = north east, align=center, double copy shadow, draw=drawColor, fill=white]
\tikzstyle{myMindmap}=[mindmap,
every node/.style={concept, minimum size=5mm, text width=5mm}, 
% every child/.style={level distance=10mm, concept color=mmcb}
level 1/.append style={level distance=10mm,sibling angle=45},
level 2/.append style={level distance=10mm,sibling angle=45},
level 3/.append style={level distance=10mm,sibling angle=45}
]
\tikzstyle{myPlace} = [place, very thick, draw=drawColor, fill=white, drop shadow]
\tikzstyle{transExpH} = [transition, very thick, draw=drawColor, fill=white, drop
shadow, minimum width=\transW, minimum height=\transH]
\tikzstyle{transExpV} = [transition, very thick, draw=drawColor, fill=white, drop
shadow, minimum width=\transH, minimum height=\transW]
\tikzstyle{transDetH} = [transition, very thick, draw=drawColor, fill=black, drop shadow, minimum width=\transW, minimum height=\transH]
\tikzstyle{transDetV} = [transition, very thick, draw=drawColor, fill=black, drop shadow, minimum width=\transH, minimum height=\transW]
\tikzstyle{pre}=[<-, very thick, >=stealth', draw=drawColor]
\tikzstyle{preN}=[<-, very thick, >=o, draw=drawColor]
\tikzstyle{post}=[->, very thick, >=stealth', draw=drawColor]
\tikzstyle{highlight}=[draw=red]
\lstdefinestyle{customJava}{
   language=Java,
   % basicstyle=\small\ttfamily\bfseries,
   basicstyle=\ttfamily,
   keywordstyle=\color{blue}\ttfamily,
   stringstyle=\color{red}\ttfamily,
   commentstyle=\color{green}\ttfamily,
   morecomment=[l][\color{magenta}]{\#},
   % breaklines=false,
   breaklines=true, breakatwhitespace=false,
   postbreak=\raisebox{0ex}[0ex][0ex]{\ensuremath{\color{red}\hookrightarrow\space}},
   frameround=fttt,
   frame=trBL,
   backgroundcolor=\color{yellow!20},
   numbers=left,
   stepnumber=1,    
   firstnumber=1,
   numberfirstline=true,
   numberstyle=\tiny\color{black!50},
   xleftmargin=2em,
   framexleftmargin=1.5em,
   % rulesepcolor=\color{gray},
   rulecolor=\color{black}
   % linewidth=8cm,
}

\lstdefinestyle{customBash}{
   language=bash,
   % basicstyle=\small\ttfamily\bfseries,
   basicstyle=\ttfamily,
   keywordstyle=\color{blue}\ttfamily,
   stringstyle=\color{red}\ttfamily,
   commentstyle=\color{green}\ttfamily,
   morecomment=[l][\color{magenta}]{\#},
   % breaklines=false,
   breaklines=true, breakatwhitespace=false,
   postbreak=\raisebox{0ex}[0ex][0ex]{\ensuremath{\color{red}\hookrightarrow\space}},
   frameround=fttt,
   frame=trBL,
   backgroundcolor=\color{green!20},
   numbers=left,
   stepnumber=1,    
   firstnumber=1,
   numberfirstline=true,
   numberstyle=\tiny\color{black!50},
   xleftmargin=2em,
   framexleftmargin=1.5em,
   % rulesepcolor=\color{gray},
   rulecolor=\color{black}
   % linewidth=8cm,
}

\lstnewenvironment{jblock}[1][]
{
  \lstset{
    style=customJava,
    #1
  }
}{}

\newcommand{\jfile}[2][]{
  \lstinputlisting[style=customJava, title={\texttt{\detokenize{#2}}}, #1]{#2}
}

\newcommand{\jj}[2][]{\lstinline[style=customJava,#1]`#2`}
  %\colorbox{codeBgColor}{
  %  \lstinline[style=customJava,#1]`#2`
  %}
%}

\lstnewenvironment{bblock}[1][]
{
  \lstset{
    style=customBash,
    #1
  }
}{}

\newcommand{\bfile}[2][]{
  \lstinputlisting[style=customBash, title={\texttt{\detokenize{#2}}}, #1]{#2}
}

\newcommand{\bb}[2][]{
  \colorbox{bashBgColor}{
    \lstinline[style=customBash,#1]`#2`
  }
}

\graphicspath{{img/}}
\lstset{inputpath=../cSrc/}

\newcommand{\me}{\mathrm{e}}
\newcommand{\C}{\texttt{C}}
\newcommand{\gcc}{\texttt{gcc}}
\newcommand{\bash}{\texttt{Bash}}

\definecolor{links}{HTML}{2A1B81}
\hypersetup{colorlinks,linkcolor=links,urlcolor=links}

\definecolor{links}{HTML}{2A1B81}
\hypersetup{colorlinks,linkcolor=,urlcolor=links}


\mode<presentation>{
  %-------------------------1
  \usetheme{Boadilla}
  \usecolortheme{beaver}
  %-------------------------1
  %-------------------------2
  %\usetheme{Goettingen}
  %\usecolortheme{sidebartab}
  %-------------------------2
  %\useoutertheme[right]{sidebar}
  %\usefonttheme{default}
  \setbeamercovered{transparent}
  %\setbeameroption{show notes on second screen=right}
  \setbeamertemplate{navigation symbols}{}
  \setbeamertemplate{footline}{}

  \bibliographystyle{abbrv}  
  %\renewcommand\bibfont{\scriptsize}
  \setbeamertemplate{bibliography item}{\textbullet}
  \setbeamertemplate{itemize item}{\checkmark}
  \setbeamertemplate{itemize subitem}{-}
  \setbeamertemplate{enumerate items}[default]
  \setbeamertemplate{sections/subsections in toc}[square]
}

\usebackgroundtemplate
{
  \begin{tikzpicture}
    \node[opacity=0.1] {\includegraphics[scale=1]{img/logoUnifi.png}};
  \end{tikzpicture}
}


\begin{document}

\title[Trijkstra]{\textbf{Trijkstra}}
\date[4 December 2015]{\flushright 4 December 2015}
\subtitle{A Dijkstra algorithm application to path planning}
\institute[Uni. Firenze]{
  \includegraphics[width=5cm]{img/logoUnifiName.eps}
}

\author[Martina Stefano]{
  \begin{center}
    \begin{tabular}{lr}
      Stefano \textsc{Martina}\\
      \href{mailto:stefano.martina@stud.unifi.it}{stefano.martina@stud.unifi.it}&
    \end{tabular}
  \end{center}
}

\titlegraphic{
  \vspace{-0.5cm}
  \tiny
  \href{http://creativecommons.org/licenses/by-sa/4.0/}{\includegraphics[width=1cm]{img/logoCC.png}}
  This work is licensed under a
  \href{http://creativecommons.org/licenses/by-sa/4.0/}{Creative
    Commons Attribution-ShareAlike 4.0 International License}.
}

\newacro{VD}{Voronoi Diagram}

\acrodefplural{VD}[VDs]{Voronoi Diagrams}

\begin{frame}[plain]
  \titlepage
\end{frame}

\section{Background}
\begin{frame}
  \frametitle{Voronoi diagrams}
  \begin{description}
  \item[Input:] A set of points in plane (or space) called
    \alert{sites}
  \item[Output:]<2-> A partition of the plane (or space) such that each
    point of a \alert{region} is nearer to a certain site respect to
    the others
  \end{description}
  \begin{columns}
    \begin{column}{0.5\textwidth}
      \begin{center}
        \includegraphics[width=\textwidth]{img/voroSites.eps}
      \end{center}
    \end{column}
    \begin{column}{0.5\textwidth}
      \begin{center}
        \includegraphics[width=\textwidth]<2->{img/voronoi.eps}
      \end{center}
    \end{column}
  \end{columns}
\end{frame}

\begin{frame}
  \frametitle{B-spline cubiche}
\end{frame}

\begin{frame}
  \frametitle{Background}
  \begin{block}{Main problem}
    \alert{Path planning} from a \alert{start} point to an \alert{end}
    point in 3D space with obstacles using \alert{Voronoi} diagrams.
  \end{block}\pause
  \begin{enumerate}
  \item Distribute \alert{points} in the surfaces of obstacles
    \begin{itemize}
    \item and optionally in the surface of bounding box\pause
    \end{itemize}
  \item Build \alert{Voronoi} diagram using those points as
    source\pause
  \item Transform the Voronoi diagram in a \alert{graph}
    \begin{itemize}
    \item cells \alert{vertexes} as \alert{nodes}
    \item cells \alert{edges} as \alert{arcs} (infinite edges
      ignored)\pause
    \end{itemize}
  \item \alert{Prune} the arcs that crosses an obstacle's
    surface\pause
  \item Attach the \alert{start} and \alert{end} points to the
    graph as nodes\pause
  \item Calculate the shortest path from start node to end node using
    \alert{Dijkstra}'s algorithm.
  \end{enumerate}
\end{frame}

\begin{frame}
  \frametitle{Example}
  \begin{center}
    \includegraphics[width=0.9\textwidth]<1>{img/screen1.eps}
    \includegraphics[width=0.9\textwidth]<2>{img/screen2.eps}
    \includegraphics[width=0.9\textwidth]<3>{img/screen3.eps}
    \includegraphics[width=0.9\textwidth]<4>{img/screen4.eps}
    \includegraphics[width=0.9\textwidth]<5>{img/screen5.eps}
  \end{center}
\end{frame}

\section{Implementation}

\begin{frame}
  \frametitle{Improvement}
  \begin{block}{Idea}
    Make a \alert{smoother} curve instead of finding the polygonal chain
    of the shortest path in the structure
  \end{block}
  \pause
  \begin{itemize}
  \item we can use a \alert{B-Spline} that \pause
    \begin{itemize}
    \item \alert{interpolate} the start and end vertexes\pause
    \item use the shortest path found with
      Dijkstra as \alert{control polygon}
    \end{itemize}
  \end{itemize}
\end{frame}

\begin{frame}
  \frametitle{Problem}  
\end{frame}

\begin{frame}
  \begin{center}
	\textbf{\calligra\Huge The End.}\\
  \includegraphics[width=5cm]{img/ornament.eps}\\[1cm]
	\pause
	{\huge\calligra Questions?\pause{} Thank you!}
  \end{center}
\end{frame}
\end{document}