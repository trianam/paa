%This work is licensed under the Creative Commons
%Attribution-ShareAlike 4.0 International License. To view a copy of
%this license, visit http://creativecommons.org/licenses/by-sa/4.0/ or
%send a letter to Creative Commons, PO Box 1866, Mountain View, CA
%94042, USA.

%\documentclass[gray,handout, pdftex, 11pt]{beamer}
%\documentclass[handout, pdftex, 11pt]{beamer}

\documentclass[pdftex, 11pt]{beamer}

\usepackage[utf8]{inputenc}
\usepackage[T1]{fontenc}
\usepackage{lmodern}
%\usepackage[italian]{babel}
\usepackage{acronym}
\usepackage{graphicx}
\usepackage{multirow}
\usepackage{listings}
\usepackage{microtype}
\usepackage{acronym}
\usepackage{array}
\usepackage{tikz}
\usetikzlibrary{shapes, chains, scopes, shadows, positioning, arrows,
  decorations.pathmorphing, calc, mindmap, petri}

\def\transW{8mm}
\def\transH{2mm}

\colorlet{mmcb}{black!20}
\colorlet{mmc1}{red!80}
\colorlet{mmc2}{blue!80}
\colorlet{c1}{green!20}
\colorlet{c2}{blue!10}
\colorlet{c3}{yellow!10}
\colorlet{c4}{red!10}
\colorlet{drawColor}{black!80}
\colorlet{commentColor}{green!70!black!90}
\colorlet{codeBgColor}{yellow!50}
\colorlet{bashBgColor}{green!50}

\tikzset{onslide/.code args={<#1>#2}{%
  \only<#1>{\pgfkeysalso{#2}} % \pgfkeysalso doesn't change the path
}}
\tikzset{temporal/.code args={<#1>#2#3#4}{%
  \temporal<#1>{\pgfkeysalso{#2}}{\pgfkeysalso{#3}}{\pgfkeysalso{#4}} % \pgfkeysalso doesn't change the path
}}

\tikzstyle{oval}=[ellipse, align=center, drop shadow, draw=drawColor, fill=white]
\tikzstyle{rect}=[rectangle, rounded corners=2pt, align=center, drop
shadow, draw=drawColor, fill=white]
\tikzstyle{comment}=[text=commentColor,font=\itshape]
\tikzstyle{textLab}=[]
\tikzstyle{arrow}=[->, very thick, >=stealth', draw=black!80]
\tikzstyle{darrow}=[->, dash pattern=on 3pt off2pt, very thick, >=stealth', draw=black!80]
\tikzstyle{fInt}=[circle, align=center, drop shadow, draw=drawColor, fill=white]
\tikzstyle{fStartEnd}=[ellipse, align=center, drop shadow, draw=drawColor, fill=white]
\tikzstyle{fInput}=[trapezium, trapezium left angle=70, trapezium right angle=110,
align=center, drop shadow, draw=drawColor, fill=white]
\tikzstyle{fProcess}=[rectangle, align=center, drop shadow, draw=drawColor, fill=white]
\tikzstyle{fSelection}=[diamond, shape aspect=3, align=center, drop
shadow, draw=drawColor, fill=white]
\tikzstyle{fOutput}=[tape, tape bend top=none, align=center, drop shadow, draw=drawColor, fill=white]
\tikzstyle{mem}=[rectangle, align=center, draw=drawColor, fill=white]
\tikzstyle{clo}=[cloud, aspect=2, align=center, drop shadow, draw=drawColor, fill=white]
\tikzstyle{file}=[chamfered rectangle, chamfered rectangle corners =
north east, align=center, drop shadow, draw=drawColor, fill=white]
\tikzstyle{files}=[chamfered rectangle, chamfered rectangle corners = north east, align=center, double copy shadow, draw=drawColor, fill=white]
\tikzstyle{myMindmap}=[mindmap,
every node/.style={concept, minimum size=5mm, text width=5mm}, 
% every child/.style={level distance=10mm, concept color=mmcb}
level 1/.append style={level distance=10mm,sibling angle=45},
level 2/.append style={level distance=10mm,sibling angle=45},
level 3/.append style={level distance=10mm,sibling angle=45}
]
\tikzstyle{myPlace} = [place, very thick, draw=drawColor, fill=white, drop shadow]
\tikzstyle{transExpH} = [transition, very thick, draw=drawColor, fill=white, drop
shadow, minimum width=\transW, minimum height=\transH]
\tikzstyle{transExpV} = [transition, very thick, draw=drawColor, fill=white, drop
shadow, minimum width=\transH, minimum height=\transW]
\tikzstyle{transDetH} = [transition, very thick, draw=drawColor, fill=black, drop shadow, minimum width=\transW, minimum height=\transH]
\tikzstyle{transDetV} = [transition, very thick, draw=drawColor, fill=black, drop shadow, minimum width=\transH, minimum height=\transW]
\tikzstyle{pre}=[<-, very thick, >=stealth', draw=drawColor]
\tikzstyle{preN}=[<-, very thick, >=o, draw=drawColor]
\tikzstyle{post}=[->, very thick, >=stealth', draw=drawColor]
\tikzstyle{highlight}=[draw=red]
\lstdefinestyle{customJava}{
   language=Java,
   % basicstyle=\small\ttfamily\bfseries,
   basicstyle=\ttfamily,
   keywordstyle=\color{blue}\ttfamily,
   stringstyle=\color{red}\ttfamily,
   commentstyle=\color{green}\ttfamily,
   morecomment=[l][\color{magenta}]{\#},
   % breaklines=false,
   breaklines=true, breakatwhitespace=false,
   postbreak=\raisebox{0ex}[0ex][0ex]{\ensuremath{\color{red}\hookrightarrow\space}},
   frameround=fttt,
   frame=trBL,
   backgroundcolor=\color{yellow!20},
   numbers=left,
   stepnumber=1,    
   firstnumber=1,
   numberfirstline=true,
   numberstyle=\tiny\color{black!50},
   xleftmargin=2em,
   framexleftmargin=1.5em,
   % rulesepcolor=\color{gray},
   rulecolor=\color{black}
   % linewidth=8cm,
}

\lstdefinestyle{customBash}{
   language=bash,
   % basicstyle=\small\ttfamily\bfseries,
   basicstyle=\ttfamily,
   keywordstyle=\color{blue}\ttfamily,
   stringstyle=\color{red}\ttfamily,
   commentstyle=\color{green}\ttfamily,
   morecomment=[l][\color{magenta}]{\#},
   % breaklines=false,
   breaklines=true, breakatwhitespace=false,
   postbreak=\raisebox{0ex}[0ex][0ex]{\ensuremath{\color{red}\hookrightarrow\space}},
   frameround=fttt,
   frame=trBL,
   backgroundcolor=\color{green!20},
   numbers=left,
   stepnumber=1,    
   firstnumber=1,
   numberfirstline=true,
   numberstyle=\tiny\color{black!50},
   xleftmargin=2em,
   framexleftmargin=1.5em,
   % rulesepcolor=\color{gray},
   rulecolor=\color{black}
   % linewidth=8cm,
}

\lstnewenvironment{jblock}[1][]
{
  \lstset{
    style=customJava,
    #1
  }
}{}

\newcommand{\jfile}[2][]{
  \lstinputlisting[style=customJava, title={\texttt{\detokenize{#2}}}, #1]{#2}
}

\newcommand{\jj}[2][]{\lstinline[style=customJava,#1]`#2`}
  %\colorbox{codeBgColor}{
  %  \lstinline[style=customJava,#1]`#2`
  %}
%}

\lstnewenvironment{bblock}[1][]
{
  \lstset{
    style=customBash,
    #1
  }
}{}

\newcommand{\bfile}[2][]{
  \lstinputlisting[style=customBash, title={\texttt{\detokenize{#2}}}, #1]{#2}
}

\newcommand{\bb}[2][]{
  \colorbox{bashBgColor}{
    \lstinline[style=customBash,#1]`#2`
  }
}

\graphicspath{{img/}}
\lstset{inputpath=../cSrc/}

\newcommand{\me}{\mathrm{e}}
\newcommand{\C}{\texttt{C}}
\newcommand{\gcc}{\texttt{gcc}}
\newcommand{\bash}{\texttt{Bash}}

\definecolor{links}{HTML}{2A1B81}
\hypersetup{colorlinks,linkcolor=links,urlcolor=links}

\definecolor{links}{HTML}{2A1B81}
\hypersetup{colorlinks,linkcolor=,urlcolor=links}


\mode<presentation>{
  %-------------------------1
  \usetheme{Boadilla}
  \usecolortheme{beaver}
  %-------------------------1
  %-------------------------2
  %\usetheme{Goettingen}
  %\usecolortheme{sidebartab}
  %-------------------------2
  %\useoutertheme[right]{sidebar}
  %\usefonttheme{default}
  \setbeamercovered{transparent}
  %\setbeameroption{show notes on second screen=right}
  \setbeamertemplate{navigation symbols}{}
  \setbeamertemplate{footline}{}

  \bibliographystyle{abbrv}  
  %\renewcommand\bibfont{\scriptsize}
  \setbeamertemplate{bibliography item}{\textbullet}
  \setbeamertemplate{itemize item}{\checkmark}
  \setbeamertemplate{itemize subitem}{-}
  \setbeamertemplate{enumerate items}[default]
  \setbeamertemplate{sections/subsections in toc}[square]
}

\usebackgroundtemplate
{
  \begin{tikzpicture}
    \node[opacity=0.1] {\includegraphics[scale=1]{img/logoUnifi.png}};
  \end{tikzpicture}
}